\subsection{Ergosterol}

The ergosterol content of the sediments in the bottles ranged from 39.02 to 68.56 $\mu$g ergosterol (g AFDM)\textsuperscript{-1} with a median of 58.87 $\mu$g ergosterol (g AFDM)\textsuperscript{-1}. This was less than the organic matter normalized ergosterol content of the leaves, which ranged between 72.88 and 203.80 $\mu$g ergosterol (g AFDM)\textsuperscript{-1} with a median of 119.20 $\mu$g ergosterol (g AFDM)\textsupersript{-1}. However, the sediments were collected in a 0.5 cm\textsuperscript{2} by 1 cm deep core which likely contains substantial organic matter below the surface that is not colonized by fungi. As a result normalizing the ergosterol content of the sample by organic matter likely under-represents the fungal density on available substrate.  To better compare the fungal abundance of the leaves and sediments, we normalized the ergosterol mass by the surface area of the leaves and the sediment surface, assuming that fungal colonization is primarily in on the surface of the leaves and in the surface sediments.

Normalized by surface area, the sediment ergosterol density ranged from 2.3 to 3.0 $\mu$g cm\textsuperscript{-2} with a median of 2.67 $\mu$g cm\textsuperscript{-2}. The median ergosterol density of the leaf discs was 0.14 $\mu$g cm\textsupersript{-2} (range = 0.08 - 0.23 $\mu$g cm\textsuperscript{-2}), which was significantly less than the ergosterol density of the sediments (Kruskal-Wallis chi-squared = 15.36, df = 1, p = 8.885e-05)(Fig. \ref{fig:erg_areal}). Nether the nutrient additions nor the leaf litter addition had an effect on the sediment ergosterol density (Fig. \ref{fig:sed_erg_by_treatment}).
