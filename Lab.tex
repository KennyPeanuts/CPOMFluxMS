\subsection{Lab Procedures}
Once back at the lab, the 1m and 0.4m sieved sediment slurry was combined and allowed to settle overnight. Overlaying water could then be siphoned off. 10ml sediment samples were taken and placed into pre-weighed crucibles, dried, and then ashed at 550 degrees Celcius. 
    The contents of the 1 Liter bottles were rinsed through a 1mm sieve and visible macroinvertebrates were removed. All retained materials were placed into pre-weighed paper bags. The bags were then placed in a drier at 50 degrees Celcius. Once dry, the CPOM was weighed and ashed at 550 degrees Celcius to determine ash free dry mass (AFDM). The AFDM was used to determine the amount of CPOM, in this case leaf disks, needed to add to the BOD bottles.

    The organic matter content of the leaf disks was determined by randomly selecting 5 senescent tulip poplar leaves that were collected from fall 2013. Each leaf was gently submerged in DI water until it was soft enough to core (about 5 min). A single leaf disk (10mm, #5 cork borer) was cut from the leaf blade avoiding the midrib. The disk was placed into a pre-weighed crucible and dried at 50 degrees Celcius, then ashed at 550 degrees Celcius. CPOM treated BOD bottles ultimately recieved 20 leaf disks.

An initial water analysis was performed on the collected water to give a nutrient baseline for the ambient replacement water. Dr. Dina Leech filtered 50 ml of the collected pond water through the GFF and utilized Hach test kits to measure nitrate, nitrite, ammonia, and orthophosphate levels in the ambient water. N was tested using Hach Test Kit N1-12 (cat# 14081-00) using the provided instructions. Detection limits were <8.8 mg/L* for  Nitrate, <0.066mg/L for Nitrite, <0.2 mg/L for Ammonia, <0.2 mg/L for Orthophosphate. The nutrient enriched water that was used for nutrient treatment BOD bottles had a target of 300 \mu m NH4NO3 and 30 \mu m of KH2PO4.

