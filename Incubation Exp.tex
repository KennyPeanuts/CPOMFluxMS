\subsection{Incubation Experiment}
Samples for the experiment were collected from Lancer Park Pond, a stormwater collection pond in Farmville, VA. The pond has a surface area of ~0.06 ha and a maximum depth of 1.5m. Field sediment samples were collected from the littoral zone at depths of 1m and 0.4m. on May 29, 2014, using an Ekman-Dredge. The sediments were sieved through a 250 µm mesh net and retained in buckets. Since, by definition, a “macroinvertebrate is any invertebrate that is 250 µm or larger,” the mesh net effectively eliminated macroinvertebrates from our sediment samples. The 1m and 0.4m sieved sediment slurry was combined and allowed to settle overnight. Overlaying water could then be siphoned off. The CPOM retained by the net was placed into 1 liter bottles. The retained CPOM would later be used to to determine the CPOM density of Lancer Park Pond and to give an estimate of the amount of CPOM to add to our BOD bottles.

Once back at the lab, the 1m and 0.4m sieved sediment slurry was combined and allowed to settle overnight. Overlaying water could then be siphoned off. 10ml sediment samples were taken and placed into pre-weighed crucible, dried, and then ashed at 550 degrees Celcius. The contents of the 1 Liter bottles were also rinsed through a 1mm sieve and visible macroinvertebrates were removed. All retained material was placed into pre-weighed paper bags. The bags were then placed in a drier at 50 degrees Celcius. Once dry, the CPOM was weighed and ashed at 550 degrees Celcius to determine ash free dry mass (AFDM). The AFDM was used to determine the amount of CPOM, in this case leaf disks, needed to add to the BOD bottles.

The organic matter content of the leaf disks was determined by randomly selecting 5 senescent tulip poplar leaves that were collected from fall 2013. Each leaf was gently submerged in DI water until it was soft enough to core (about 5 min). A single leaf disk (10mm, #5 cork borer) was cut from the leaf blade avoiding the midrib. The disk was placed into a pre-weighed crucible and dried at 50 degrees Celcius, then ashed at 550 degrees Celcius. CPOM treatmets utimately contained 20 leaf disks.

An initial water analysis was performed on the water collected to give a nutrient baseline for the ambient replacement water. Dr. Dina Leech filtered 50 ml of the collected pond water through the GFF and utilized Hach test kits to measure nitrate, nitrite, ammonia, and orthophosphate levels in the ambient water. N was tested using Hach Test Kit N1-12 (cat# 14081-00) using the provided instructions. Detection limits were <8.8 mg/L* for  Nitrate, <0.066mg/L for Nitrite, <0.2 mg/L for Ammonia, <0.2 mg/L for Orthophosphate. The nutrient enriched water that was used for nutrient treatment BOD bottles had a target of 300 \mu m NH4NO3 and 30 \mu m of KH2PO4.

* This number needs to be corrected.

The BOD bottles were incubated in the dark, at room temperature, on rocker-shakers at speed 8 and tilt 8. To encourage gas exchange in the bottles, the bottles were stored with 15ml of water removed. Thus, the total water was approximately 285ml. At each sampling, the BOD bottles were removed from incubation, and then approximately 68ml of water were drawn from the bottle. 2, 15 ml samples were collected with a glass syringe and placed in 10 ml vials. One vial was sealed and then fixed for Time 0 (T-0). The other sample was placed on the rocker-shaker to incubate in the dark for five hours. The final water collected was for bacterial, nutrient, and absorbence samples. The 3ml  bacterial samples were stored in a bacterial vial, preserved in gluteraldehyde in the fridge. The absorbance samples were stored in the refrigerator, while the nutrient samples were kept in the freezer. Bottles were set-up on 9 June 2014 and sampled on days 1 (10 June 2014),3, 8, 15, and 22. SOD was calculated as the change in dissolved oxygen over time. Dissolved oxygen was determined using the Winkler's Titration method. 

At the conclusion of the experiment, a sample for sediment C:N ratio and a sediment ergosterol sample were taken from each of the BOD bottles. In addition, in BOD bottles with CPOM, leaf disks were taken for C:N ratio samples and leaf ergosterol samples. 
