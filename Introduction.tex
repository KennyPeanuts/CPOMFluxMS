\section{Introduction}
Ecosystem subsidies, the movement of resources across ecosystem boundries \cite{Polis_1997}, are an important part of organic matter cycling in aquatic systems. The reciprical transfer of resources between aquatic and terrestrial systems is common \cite{Nakano_2001}, however the input of terrestrial organic matter to aquatic systems is an especially significant flux of material since, this subsidy has been shown to support metabolism and secondary production in a majority of lentic and lotic ecosystems\cite{Marcarelli_2011}. Organic matter subsidies from terrestrial to aquatic ecosystems are dominated by detrital plant material either as dissolved (DOC) or particulate (POC) organic carbon, and can substantially augment autochthonous organic matter production \cite{Hodkinson_1975,GASITH_1976,wetzel_1984,WETZEL_1995,Webster_1997,Kobayashi_2011,Mehring_2014}.   

The direct input of DOC dominates terrestrial subsidies in most aquatic systems \cite{Rich_1978,wetzel_1984,CITE} but POC inputs, mainly in the form of leaf litter, can substantially augment aquatic organic matter pools \cite{Wetzel_1972,Hodkinson_1975,GASITH_1976,Rich_1978,Wallace_1999,Mehring_2014}. During the process of leaf litter decomposition in aquatic systems, the leaf biomass supplies distinct subsidies to the aquatic ecosystem \cite{Gessner_1999,Marcarelli_2011}. Up to 30\% of the initial mass of leaves can be leached as DOC \cite{CITE,Meyer_1998,Duan_2014}, although large initial DOC fluxes from dried leaves in decomposition experiments may be an artifact of air drying the leaves (CITE). This supply of DOC is an important component of aquatic organic matter budgets \cite{McDowell_1976,Karlsson_2007} and has been shown to alter the abundance \cite{Bott_1984,Fey_2015} and function \cite{MCCONNELL_1968,Lennon_2005} of aquatic microbial communities. Furthermore, DOC subsidies processed through the microbial loop support metazoan production \cite{Hall_1998,Wilcox_2005,Fey_2015b}.  The leaf mass that remains following leaching can be transferred directly to the biomass of aquatic invertebrate \cite{Wallace_1999,Kobayashi_2011} and vertebrate \cite{Rubbo_2008} consumers via consumption of the leaf material or act as substrate for bacterial and fungal production \cite{Gessner_1999}. 

The majority of research on leaf litter additions to aquatic systems has focused on its impact on microbial and metazoan production, however the process of leaf litter decomposition also alters the chemical and physical environment of aquatic systems \cite{Gessner_1999}. Leaf leachates reduce light penetration \cite{CITE} and alter pH \cite{CITE}. Furthermore leachates provide bioavailable organic nutrients \cite{McConnell_1968,Duan_2014} and have been shown to increase total phosphorus \cite{Feh_2015b}, and total nitrogen \cite{Feh_2015} concentration in overlying water. Mineralization of leaf organic matter by microbial or animal consumers, results in the release of inorganic nutrients and CO\textsubscript{2} \cite{CITE} from the leaf mass. Typically, however, the stoichometric imbalance between microbial consumers and detritus means that leaves are sites of net immobilization of inorganic nitrogen and phosphorus \cite{CITE}. The mineralization of organic carbon in the leaves creates a demand for oxygen \cite{CITE} that can lower dissolved oxygen concentrations in water overlying decomposing leaf litter \cite{Hodkinson_1975,Rubbo_2008,Mehring_2014,Feh_2015b}.

Although leaf litter represents an important subsidy in both lentic and lotic systems \cite{Webster_1986}, the physical differences between these systems will likely alter the specific effects of leaf litter on ecosystem function. The redistribution of sediments and other materials due to the flowing water in lotic systems tends to be more variable and extensive than in lentic systems \cite{Wetzel_2001}, homogenizing the chemical and physical gradients produced by leaf litter decomposition over greater spatial extent.  Lakes, ponds, and reservoirs, on the other hand often possess substantial physical and chemical gradients over relatively small spatial extents. The thermal stratification preset in many lentic systems results in the creation of distinct chemical and physical habitats within a lake or pond, further the lack of the strong advective currents found in lotic systems means that lake processes, particularly in the sediments, are influenced by material diffusion rates (\ref{CITE}). Lentic systems are sites of organic matter production, mineralization, and storage \cite{Tranvik_2009} and all of these processes are sensitive to the physical and chemical environment in the lake. Alteration of the physical and chemical environment of lakes as a result of leaf decomposition has the potential to affect the biogeochemical role of lentic systems on the landscape. 

Within lentic systems, there are important differences between natural and man--made systems, particularly for the smallest lakes and ponds. The abundance of small (< 0.1 km\textsuperscript{2}) lakes is greater than 2 orders of magnitude greater than even lakes with a surface area of 1 km\textsuperscript{2} \cite{Downing_2010} and the cumulative surface area small lakes is essentially equal to that of largest lakes in the world.  

Despite the significance of leaf litter subsidies to aquatic systems there remains considerable gaps in our knowledge of the effects of litter decay in lentic systems, particularly small man--made ponds.  There are very few estimates of the density of leaf litter in small lentic systems (but see \ref{CITE}), so one of our objectives was to quantify the density of leaf litter in small man--made ponds. The rate of leaf litter decomposition in lentic systems has been shown to be overall lower than for lotic systems (\ref{WebsterandBenfield, OTHERS}) but man--made ponds remain under studied and thus our second objective was to measure the rate of decomposition for standard leaf pack subsidies in ponds of contrasting construction and hydrology.  

Our study had 3 main objectives: 1) quantify the density leaf litter in the sediments of small man-made ponds, 2) measure the decomposition of standardized leaf pack subsidies to man--made ponds of contrasting construction and hydrology, and 3) evaluate the effect that leaf litter subsidies have on nutrient cycling and sediment oxygen demand in pond sediments. We hypothesize that leaf litter subsidies result in leaf litter being an abundant and persistent component of the sediment organic matter in man--made ponds, and the presence of leaf litter in the sediments will increase sediment oxygen demand, increase DOC concentration and bioavaliablity, and decrease the flux of DIN and DIP from the sediments. We further hypothesize that sediments containing leaf litter with have greater fungal biomass and sediment organic matter than sediments without leaf litter.