\section{Introduction}
Ecosystem subsidies, the transfer of resources across ecosystem boundries \cite{Polis_1997}, are an important part of organic matter cycling in aquatic systems. Although the reciprical transfer of resources between aquatic and terrestrial systems is common (cite), the majority of lentic and lotic ecosystems rely heavily on terrestrial organic matter inputs to support their metabolism and secondary production \cite{Marcarelli_2011}. Organic matter subsidies from terrestrial to aquatic ecosystems are dominated by detrital plant material either as dissolved (DOC) or particulate (POC) organic carbon. Although DOC and POC inputs are variable across systems, they can represent a substantial augmentation of autochthonous organic matter production \cite{Hodkinson_1975,GASITH_1976,wetzel_1984,WETZEL_1995,Webster_1997, Kobayashi_2011, Mehring_2014}.   

The food webs of forested streams are heavily dependent on the input of leaf litter (POC) to support secondary production \cite{wallaceetal_1999}  but even streams with high autochthonous production \cite{Mineau_2012} and urbanized streams \cite{Duan_2014} respond to alterations in leaf litter inputs. Dissloved organic carbon inputs into streams have received less attention than POC subsidies but DOC subsidies are an important component of stream organic matter budgets \cite{McDowell_1976} and have been shown to alter the abundance \cite{Bott_1984} and function \cite{Bernhardt_2002} of stream microbial communities. Furthermore, DOC subsidies processed through the microbial loop can support metazoan production \cite{Hall_1998,Wilcox_2005}.

In lentic systems, 



%Allochthonous organic matter inputs are an important basal resource in many aquatic systems. Terrestrial leaf litter inputs determine the biogeochemistry and trophic dynamics of small streams \cite{websterandbenfield1986,wallaceetal1997,wallaceetal1999} and recently it has been shown that temperate lake food webs depend on dissolved allochthonous organic matter \ref{paceetal#####}. However, lakes also receive substantial inputs of terrestrial leaf litter \ref{hodkinson1975, reed1979, wallaceandbenfield1986, franceandpeters1995, kobayashietal2011}, and the presence of litter can alter lentic nutrient cycling \ref{wetzel} and community structure \ref{}. 

%Although this has not been extensively studied, the distribution of terrestrial leaf litter tends to be restricted to near the littoral portions of the lake \ref{} suggesting that leaf litter will have a greater impact on smaller lakes.  

%This study evaluated the distribution and breakdown of terrestrial leaf litter in small man--made ponds in central Virginia. We hypothesized that leaf litter would be present and persistent in the pond sediments.  We further evaluated the impact that terrestrial leaf litter had on the 
