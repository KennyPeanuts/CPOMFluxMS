\section{Introduction}

Allochthonous organic matter inputs are an important basal resource in many aquatic systems. Terrestrial leaf litter inputs determine the biogeochemistry and trophic dynamics of small streams \cite{websterandbenfield1986, wallaceetal1997, wallaceetal1999} and recently it has been shown that temperate lake food webs depend on dissolved allochthonous organic matter \ref{paceetal#####}. However, lakes also receive substantial inputs of terrestrial leaf litter \ref{hodkinson1975, reed1979, wallaceandbenfield1986, franceandpeters1995, kobayashietal2011}, and the presence of litter can alter lentic nutrient cycling \ref{wetzel} and community structure \ref{}. 

Although this has not been extensively studied, the distribution of terrestrial leaf litter tends to be restricted to near the littoral portions of the lake \ref{} suggesting that leaf litter will have a greater impact on smaller lakes.  

This study evaluated the distribution and breakdown of terrestrial leaf litter in small man--made ponds in central Virginia. We hypothesized that leaf litter would be present and persistent in the pond sediments.  We further evaluated the impact that terrestrial leaf litter had on the 
