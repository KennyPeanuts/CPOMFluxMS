\section{Introduction}
Ecosystem subsidies, the movement of resources across ecosystem boundries \cite{Polis_1997}, are an important part of organic matter cycling in aquatic systems. Although the reciprical transfer of resources between aquatic and terrestrial systems is common \cite{Nakano_2001}, the majority of lentic and lotic ecosystems appear to rely heavily on terrestrial organic matter inputs to support their metabolism and secondary production \cite{Marcarelli_2011}. Organic matter subsidies from terrestrial to aquatic ecosystems are dominated by detrital plant material either as dissolved (DOC) or particulate (POC) organic carbon, which can represent a substantial augmentation of autochthonous organic matter production \cite{Hodkinson_1975,GASITH_1976,wetzel_1984,WETZEL_1995,Webster_1997,Kobayashi_2011,Mehring_2014}.   

The direct input of DOC dominates terrestrial subsidies in many aquatic systems \cite{Rich_1978,wetzel_1984,CITE} but POC inputs, mainly in the form of leaf litter, can substantially augment aquatic organic matter pools \cite{Wetzel_1972,Hodkinson_1975,GASITH_1976,Rich_1978,Wallace_1999,Mehring_2014}. During the process of leaf litter decomposition in aquatic systems, leaves supply distinct subsidies to aquatic systems \cite{Gessner_1999,Marcarelli_2011}. Up to 30\% of the initial mass of leaves can be leached as DOC \cite{CITE,Meyer_1998,Duan_2014}, although large initial DOC fluxes from dried leaves may be an artifact of the drying process (CITE). This supply of DOC is an important component of aquatic organic matter budgets \cite{McDowell_1976,Karlsson_2007} and has been shown to alter the abundance \cite{Bott_1984,Fey_2015} and function \cite{MCCONNELL_1968,Bernhardt_2002} of aquatic microbial communities. Furthermore, DOC subsidies processed through the microbial loop can support metazoan production \cite{Hall_1998,Wilcox_2005,Fey_2015b}.  The unleached leaf mass can be transferred directly to the biomass of aquatic invertebrate \cite{Kobayashi} and vertebrate \cite{Rubbo_2008} consumers via consumption of the leaf material \cite{Gessner_1999}. Following a 3-year litter exclusion, \cite{Wallace_1999} found a significant reduction in the biomass of stream invertebrate consumers. 

In addition to supporting microbial and metazoan production, the process of leaf litter decomposition can alter the chemical and physical environment of aquatic systems \cite{Gessner_1999}. Leaf leachates provide bioavailable organic nutrients \cite{McConnell_1968,Duan_2014} and have been shown to increase total phosphorus \cite{Feh_2015b}, and total nitrogen \cite{Feh_2015} concentration in overlying water. Mineralization of leaf organic matter by microbial or animal consumers, will result in the release of inorganic nutrients and CO\textsubscript{2} \cite{CITE}. Typically, however, the stoichometric imbalance between microbial consumers and detritus means that leaves are sites of net immobilization of inorganic nitrogen and phosphorus \cite{CITE}. The mineralization of organic carbon in the leaves creates a demand for oxygen \cite{CITE} that can lower dissolved oxygen concentrations in water overlying decomposing leaf litter \cite{Hodkinson_1975,Rubbo_2008,Mehring_2014,Feh_2015b}.

In this study we evaluate the effect that leaf litter subsidies have on nutrient cycling and sediment oxygen demand in the sediments of a small man--made pond. We hypothesize that leaf litter will 1) increase sediment oxygen demand, 2) increase DOC concentration and bioavaliablity, and 3) decrease the flux of DIN and DIP from the sediments. We further hypothesize that sediments with leaf litter with have greater fungal biomass and sediment organic matter than sediments without leaf litter.