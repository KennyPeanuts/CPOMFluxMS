\section{Introduction}
Ecosystem subsidies, the movement of resources across ecosystem boundries \cite{Polis_1997}, are an important part of organic matter cycling in aquatic systems. Although the reciprical transfer of resources between aquatic and terrestrial systems is common \cite{Nakano_2001}, the majority of lentic and lotic ecosystems rely heavily on terrestrial organic matter inputs to support their metabolism and secondary production \cite{Marcarelli_2011}. Organic matter subsidies from terrestrial to aquatic ecosystems are dominated by detrital plant material either as dissolved (DOC) or particulate (POC) organic carbon, which can represent a substantial augmentation of autochthonous organic matter production \cite{Hodkinson_1975,GASITH_1976,wetzel_1984,WETZEL_1995,Webster_1997,Kobayashi_2011,Mehring_2014}.   

The direct input of DOC can dominate the terrestrial subsidies in many aquatic systems \cite{Rich_1978,wetzel_1984,CITE} but POC inputs, mainly in the form of leaf litter can substantially augment aquatic organic matter pools \cite{Wetzel_1972,Hodkinson_1975,GASITH_1976,Rich_1978,Wallace_1999,Mehring_2014}. During the process of leaf litter decomposition in aquatic systems, leaves supply distinct subsidies to aquatic systems \cite{Gessner_1999,Marcarelli_2011}. Up to 30\% of the initial mass of leaves can be leached as DOC \cite{CITE,Meyer_1998,Duan_2014}, although large initial DOC fluxes from dried leaves may be an artifact of the drying process (CITE). This supply of DOC is an important component of aquatic organic matter budgets \cite{McDowell_1976,Karlsson_2007} and has been shown to alter the abundance \cite{Bott_1984,Fey_2015} and function \cite{MCCONNELL_1968,Bernhardt_2002} of aquatic microbial communities. Furthermore, DOC subsidies processed through the microbial loop can support metazoan production \cite{Hall_1998,Wilcox_2005,Fey_2015b}.  Leaf mass can be transferred directly to the biomass of shredding animals via consumption of the leaf material \cite{Gessner_1999}. The biomass of aquatic invertebrate \cite{Kobayashi} and vertebrate \cite{Rubbo_2008} consumers. Following a 3-year litter exclusion, \cite{Wallace_1999} found a significant reduction in the biomass of stream invertebrate consumers. 

During decomposition, leaf litter can alter the chemical and physical environment of aquatic substrata. Mineralization of leaf organic matter by microbial or animal consumers, will result in the release of inorganic nutrients and CO\textsubscript{2} \cite{CITE}. NET FLUX - IMMOBILIZATION vs PRODUCTION. 



%The most well described examples of the way in which terrestrial POC subsidies support aquatic food webs comes from forested streams, where the input of leaf litter (POC) maintain secondary production \cite{wallaceetal_1999}  but even streams with high autochthonous production \cite{Mineau_2012} and urbanized streams \cite{Duan_2014} respond to alterations in leaf litter inputs. Dissolved organic carbon inputs into streams have received less attention than POC subsidies but DOC subsidies are an important component of stream organic matter budgets \cite{McDowell_1976} and have been shown to alter the abundance \cite{Bott_1984} and function \cite{Bernhardt_2002} of stream microbial communities. Furthermore, DOC subsidies processed through the microbial loop can support metazoan production \cite{Hall_1998,Wilcox_2005}.

%The importance of subsidies of terrestrial plant detritus for lake food webs is also well established. Since lentic systems are generally accretive \cite{WETZEL_2001}, they accumulate a substantial mass of refractory detrital DOC and POC from the watershed \cite{Rich_1978,wetzel_1984,WETZEL_1995}. This persistent supply of organic matter tends to supply heterotrophic respiration in excess of gross primary production (GPP), resulting in net heterotrophy in most lentic systems \cite{Cole_2000,Marcarelli_2011}. The importance of terrestrial organic matter subsidies to lentic systems has emphasized the support of pelagic food webs by terrestrially-derived DOC \cite{Carpenter_2005} \cite{Cole_2006} \cite{Pace_2004,Cottingham_2013,Fey_2015}. In these studies, DOC imported from the watershed is transferred to the metazoan food web through microbial utilization (CITE). Simultaneously, increases in microbial metabolism results in increases in organic matter mineralization and CO\textsubscript{2} production (CITE). 

Although recent work has highlighted the connection between terrestrial subsidies and lake pelagia, most of the lakes in the world are small \cite{Downing_2007} with significant benthic pelagic coupling (CITE). \cite{MCCONNELL_1968}





%Allochthonous organic matter inputs are an important basal resource in many aquatic systems. Terrestrial leaf litter inputs determine the biogeochemistry and trophic dynamics of small streams \cite{websterandbenfield1986,wallaceetal1997,wallaceetal1999} and recently it has been shown that temperate lake food webs depend on dissolved allochthonous organic matter \ref{paceetal#####}. However, lakes also receive substantial inputs of terrestrial leaf litter \ref{hodkinson1975, reed1979, wallaceandbenfield1986, franceandpeters1995, kobayashietal2011}, and the presence of litter can alter lentic nutrient cycling \ref{wetzel} and community structure \ref{}. 

%Although this has not been extensively studied, the distribution of terrestrial leaf litter tends to be restricted to near the littoral portions of the lake \ref{} suggesting that leaf litter will have a greater impact on smaller lakes.  

%This study evaluated the distribution and breakdown of terrestrial leaf litter in small man--made ponds in central Virginia. We hypothesized that leaf litter would be present and persistent in the pond sediments.  We further evaluated the impact that terrestrial leaf litter had on the 
