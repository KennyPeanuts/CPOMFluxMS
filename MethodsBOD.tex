\subsection{Microcosm Sampling}
At the beginning of each sampling event, an individual BOD bottle was removed from the incubation and the following samples were removed from near the midpoint of the water column with a 30 ml glass syringe and a stainless steel cannula. To determine the initial oxygen content of the water a 15 ml sample was removed and added to a 10 ml serum vial at the base of the vial. This approach allows the 5 ml of sample to overflow the bottle and limits oxygen contamination of the sample from the atmosphere. This vial was immediately fixed with for oxygen determination using Winkler titration \cite{CARPENTER_1965} modified for the small volumes. A second 15 ml sample was then removed from the bottle and added to a 10 ml serum vial using the same technique. This bottle was sealed without a headspace and incubated in the dark for five hours to determine the oxygen consumption of the water. To determine the bacterial abundance of the overlying water, a 3 ml sample was removed and placed into a sterile plastic tube and preserved with 20 $\mu$l of CONC gluteraldehyde. The results of the bacterial abundance are not reported in this paper. A 30 ml sample was then removed and filtered through a GFF filter into an acid--washed 50 ml plastic conical tube, that was frozen to determine the nutrient concentration. Finally a 10 ml sample was removed and filtered through the same GFF filter into a 15 ml plastic conical tube to determine an absorbance profile. 

Each sampling removed 73 ml of water from the 185 ml of overlying water that was in each microcosm during the incubation. To determine the total consumption of oxygen in the microcosm, we gently added 90 ml of either pond water or pond water enriched with DIN and DIP into the BOD bottle and then stoppered the bottle. The stoppered bottles were placed back onto the rocker--shakers in the dark for approximately 5 h. Following this incubation period, a  bacterial, nutrient, and absorbence samples. The 3ml  bacterial samples were stored in a bacterial vial, preserved in gluteraldehyde in the fridge. The absorbance samples were stored in the refrigerator, while the nutrient samples were kept in the freezer. Bottles were set-up on 9 June 2014 and sampled on days 1 (10 June 2014),3, 8, 15, and 22. SOD was calculated as the change in dissolved oxygen over time. Dissolved oxygen was determined using the Winkler's Titration method. 

At the conclusion of the experiment, a sample for sediment C:N ratio and a sediment ergosterol sample were taken from each of the BOD bottles. In addition, in BOD bottles with CPOM, leaf disks were taken for C:N ratio samples and leaf ergosterol samples. 