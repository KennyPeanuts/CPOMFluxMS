\subsection{Microcosm Sampling}
The total incubation of the microcosms in all the experiments consisted of sampling events and intervening incubation. During the incubation the BOD bottles were un--stoppered and had 15 ml of overlying water removed to facilitate gas exchange. In the litter experiment and leached litter experiment, the microcosms were incubated in the dark, at room temperature and agitated gently on a rocker-shaker. In the leached litter SOD experiment, the microcosms were incubated in the dark at 15.5\textsuperscript{o} C in a INCUBATOR.

At the beginning of each sampling event in the litter experiment, an individual BOD bottle was removed from the incubation and the following samples were removed from near the midpoint of the water column with a 30 ml glass syringe and a stainless steel cannula. To determine the initial oxygen content of the water a 15 ml sample was removed and added to a 10 ml serum vial at the base of the vial. This approach allows the 5 ml of sample to overflow the bottle and limits oxygen contamination of the sample from the atmosphere. This vial was immediately fixed with for oxygen determination using Winkler titration \cite{CARPENTER_1965} modified for the small volumes. A second 15 ml sample was then removed from the bottle and added to a 10 ml serum vial using the same technique. This bottle was sealed without a headspace and incubated in the dark for five hours to determine the oxygen consumption of the water. To determine the bacterial abundance of the overlying water, a 3 ml sample was removed and placed into a sterile plastic tube and preserved with 20 $\mu$l of CONC gluteraldehyde. The results of the bacterial abundance are not reported in this paper. A 30 ml sample was then removed and filtered through a GFF filter into an acid--washed 50 ml plastic conical tube, that was frozen to determine the nutrient concentration. Finally a 5 ml sample was removed and filtered through the same GFF filter into a 15 ml plastic conical tube to determine an absorbance profile. Each sampling during the litter experiment removed 53 ml of water from the 185 ml of overlying water that was in each microcosm during the incubation. 

The sampling of the leached litter experiment was identical to the litter experiment except that the samples for bacterial abundance and nutrients were not removed from the microcosms, so each sampling of the leached litter experiment removed only 35 ml of the 185 ml of overlying water. We only measured SOD during the leached litter SOD experiment, so only the 15 ml samples for SOD were removed from the 185 ml of overlying water.

To determine the total consumption of oxygen in the microcosm, we added sufficient pond water to the BOD bottle to remove the headspace when stoppered. In the case of the nutrient enriched treatments of the litter experiment, this replacement water was the pond water enriched with DIN and DIP. The stoppered bottles were placed back onto the rocker--shakers in the dark for approximately 5 h. Following this incubation period, the bottles were removed from the incubation and 15 ml of water was removed and added to a 10 ml serum vial to measure the oxygen concentration as described above. The change in oxygen concentration was corrected for the oxygen added with the pond water used to fill the bottles and then normalized by the surface area of the sediment and the duration of the incubation to calculate the sediment oxygen demand (SOD). Following this we measured the oxygen concentration (as described above) of the serum vials that had been incubated to measure the water oxygen consumption. The water oxygen consumption was very low and so this value was not subtracted from the sediment oxygen demand during the calculations. The microcosms of the litter experiment were set--up on 9 June 2014 and sampled on days 1, 3, 8, 15, and 22. The microcosms of the leached litter experiment were set--up on 22 September 2015 and sampled on days 1, 2, 9 16, and 30. The microcosms of the leached litter SOD experiment were set--up on 9 February 2016 and sampled on days 2 and 10.
 
At the conclusion of the experiment (day 23), the overlying water was siphoned from each microcosm and sediment LOI, sediment C:N, and sediment ergosterol content were each measured from a separate 1 cm X 0.8 cm sediment core.  From the microcosms containing leaf discs we randomly selected 12 leaf discs to measure LOI, 4 leaf discs to measure C:N, and 4 leaf discs to measure ergosterol abundance. Loss on ignition of the sediment and leaves was measured as described above. The sediment cores and leaf discs for C:N were dried at 50\textsuperscript{o} C and the cores and leaf discs for ergosterol abundance were immediately added to 10 ml of methanol, sealed and placed in TEMPERATURE. Carbon and nitrogen mass were measured using a Costech CHN Analyzer. Samples were acidified in HCl vapors for 24 hours prior to analysis to remove inorganic carbon.  Ergosterol was measured METHODS and normalized to the AFDM of the samples and the surface are of the core or leaf disc.  