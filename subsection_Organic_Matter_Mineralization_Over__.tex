\subsection{Organic Matter Mineralization}

Over the course of the 21 days of incubation the total oxygen consumption of the bottles ranged between 360 and 2100 $\mu mols$ O\textsubscript{2} m\textsuperscript{-2} h\textsuperscript{-1}, with a median of 920 $\mu mols$ O\textsubscript{2} m\textsuperscript{-2} h\textsuperscript{-1}. Since the treatments with added leaf litter had slightly more organic matter in the sediments than those without leaf litter, we normalized the oxygen flux to the estimated organic matter content of the sediments in the bottles. The organic matter normalized oxygen flux ranged from 7.34 to 42.54 $\mu mols$ O\textsubscript{2} (g organic matter)\textsuperscript{-1} h\textsuperscript{-1}, with a median flux of 18.97 $\mu mols$ O\textsubscript{2} (g organic matter)\textsuperscript{-1} h\textsuperscript{-1}.

The oxygen consumption of the overlying water alone ranged from -3.5 to 3.1 $\mu mols$ L\textsuperscript{-1} h\textsuperscript{-1}, where the negative flux indicates that the oxygen concentration of the water increased during the incubation.  The increases in oxygen concentration occurred primarily during the incubations on the initial and fourteenth day of the experiment (Fig. \ref{fig:waterR}). During the initial incubation the increases in oxygen occurred only in the water that came from bottles containing leaf litter (Fig. \ref{fig:waterR}).



