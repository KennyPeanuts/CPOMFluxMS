\subsection{Organic Matter Mineralization}

Over the course of the 21 days of incubation the total oxygen consumption of the bottles ranged between 360 and 2100 $\mu mols$ O\textsubscript{2} m\textsuperscript{-2} h\textsuperscript{-1}, with a median flux of 920 $\mu mols$ O\textsubscript{2} m\textsuperscript{-2} h\textsuperscript{-1}. Since the treatments with added leaf litter had slightly more organic matter in the sediments than those without leaf litter, we normalized the oxygen flux to the estimated organic matter content of the sediments in the bottles. The organic matter normalized oxygen flux ranged from 7.34 to 42.54 $\mu mols$ O\textsubscript{2} (g organic matter)\textsuperscript{-1} h\textsuperscript{-1}, with a median flux of 18.97 $\mu mols$ O\textsubscript{2} (g organic matter)\textsuperscript{-1} h\textsuperscript{-1}.

The linear mixed model evaluating the effect of days of incubation, leaf litter presence, and nutrient enrichment on the organic matter normalized oxygen flux had a REML criteria at convergence of 521.6211. The ANOVA on the model results found that the only factor that affected organic matter normalized oxygen flux was the presence of leaf litter in the bottles. Bottles containing leaf litter had a mean oxygen flux of 23.85 $\mu mols$ O\textsubscript{2} (g organic matter)\textsuperscript{-1} h\textsuperscript{-1}, which was significantly greater than the mean oxygen flux of 17.67 $\mu mols$ O\textsubscript{2} (g organic matter)\textsuperscript{-1} h\textsuperscript{-1} in the bottles without leaf litter (p = 0.0005; Fig. \ref{fig:sod}).  The patterns in Fig. \ref{fig:sod} suggest that the differences in oxygen flux between the may have disappeared by the end of the experiment, but there was no significant interaction between the leaf litter and the days of incubation (p = 0.135).

The oxygen consumption of the overlying water alone ranged from -3.5 to 3.1 $\mu mols$ L\textsuperscript{-1} h\textsuperscript{-1}, where the negative flux indicates that the oxygen concentration of the water increased during the incubation.  The increases in oxygen concentration occurred primarily during the incubations on the initial and fourteenth day of the experiment (Fig. \ref{fig:waterR}). During the initial incubation the increases in oxygen occurred only in the water that came from bottles containing leaf litter (Fig. \ref{fig:waterR}).



