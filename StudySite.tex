\section{Methods}
\subsection{Study Site}
All of the ponds used in the study are located near Farmville, VA (37.301 N, -78.396 W) (fig. \ref{fig:map}) and are small man--made ponds (Table \ref{tab:study_ponds}). Lancer Park Pond is a SIZE ha in--line pond with an earth dam and a maximum depth of 1.5 m. Lancer Park Pond has a permanent inlet and is almost completely surrounded by second growth forest. Daulton Pond is a SIZE ha headwater pond with a earth dam and a maximum depth of 3.2 m. Daulton Pond does not have a permanent inlet and is likely partially spring--fed. The riparian zone of Daulton Pond is approximately 50\% second growth forest and 50\% mowed grass. The littoral zone of Daulton Pond is mostly covered in an unidentified reed and cattails (\emph{Typha sp.}). Woodland Court Pond is a SIZE ha pond with an earth dam that is drained by a stand--pipe. The pond has a permanent and a maximum depth of 2 m and a riparan zone that is about 30\% second growth forest. The remaining portion of the riparian zone is minimally landscaped disturbed land associated with an apartment complex. Approximately 50\% of the littoral zone of Woodland Court Pond is a patch of cattail (\emph{Typha sp.}). Campus Pond is a SIZE ha constructed stormwater pond with a permanent inlet that is drained by a stand--pipe. Campus Pond has a maximum depth of 0.5 m but the basin is enclosed by a concrete wall, so it has no natural littoral zone and is nearly uniform in depth. Campus Pond is surrounded by landscaping that consists of small trees and mowed grass. Wilkes Lake is a SIZE ha man--made pond with a maximum depth of 2 m. Wilkes Lake has a permanent inlet that drains a wetland and a stand--pipe. Approximately 90\% of the lake shoreline is second growth forest and the remaining area is mowed grass.


    
