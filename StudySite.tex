\section{Methods}
\subsection{Study Site}
All of the ponds in the study are located in the vicinity of Farmville, VA (37.301 N, -78.396 W).  All of the ponds are small, man--made ponds with earthen dams (Table \ref{tab:study_ponds}).  The ponds are all relatively similar but do have some notable distinctions.  Lancer Park Pond is a 0.06 ha earth dam pond with a maximum depth of 1.5 m. Lancer Park Pond has a permanent inlet and is almost completely surrounded by second growth forest.  Daulton Pond is a 0.6 ha earth dam pond with a maximum depth of 3.2 m. Daulton Pond does not have a permanent inlet and is surrounded by second growth forest and a hay field. Additionally Daulton Pond is partially fringed by a population of cattails (\em{Typha sp.}). Woodland Court Pond is a 0.3 ha pond with an earth dam that is drained by 


Samples for the experiment were collected from Lancer Park Pond, a stormwater collection pond in Farmville, VA. The pond has a surface area of ~0.06 ha and a maximum depth of 1.5m. Field sediment samples were collected from the littoral zone on May 29, 2014, using an Ekman-Dredge. Half of the sediments were collected from a depth of 0.4m and the other half were collected from 1m depth. The sediments were sieved through a 250 µm mesh net and retained in buckets. Since, by definition, a “macroinvertebrate is any invertebrate that is 250 µm or larger,” the mesh net effectively eliminated macroinvertebrates from our sediment samples. The CPOM retained by the net was placed into 1 liter bottles. The retained CPOM would later be used to to determine the CPOM density of Lancer Park Pond and to give an estimate of the amount of CPOM to add to our BOD bottles. Water was collected from the pond at a depth of 0.5m on June 9, 2014.
    
