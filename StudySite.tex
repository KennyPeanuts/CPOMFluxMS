\section{Methods}
\subsection{Study Site}
All of the ponds used in the study are located in the vicinity of Farmville, VA (37.301 N, -78.396 W) and are small man--made ponds (Table \ref{tab:study_ponds}).  Lancer Park Pond is a 0.06 ha in--line pond with an earth dam and a maximum depth of 1.5 m. Lancer Park Pond has a permanent inlet and is almost completely surrounded by second growth forest. Daulton Pond is a 0.6 ha headwater pond with a earth dam and a maximum depth of 3.2 m. Daulton Pond does not have a permanent inlet and is likely partially spring--fed. The riparian zone of Daulton Pond is approximately 50\% second growth forest and 50\% mowed grass. Daulton Pond is the only pond in the study that has a substantial community of macrophytes. The pond is partially fringed by a population of cattails (\em{Typha sp.}) that cover approximately 30\% of its littoral zone, and a bed of MACROPHYTE along its southern shore. Woodland Court Pond is a 0.3 ha pond with an earth dam that is drained by a stand--pipe. The pond has a maximum depth of INSERT DEPTH and a riparan zone that is about 30\% second growth forest. The remaining portion of the riparian zone is minimally landscaped disturbed land associated with an apartment complex. Wilkes Lake is a 


Samples for the experiment were collected from Lancer Park Pond, a stormwater collection pond in Farmville, VA. The pond has a surface area of ~0.06 ha and a maximum depth of 1.5m. Field sediment samples were collected from the littoral zone on May 29, 2014, using an Ekman-Dredge. Half of the sediments were collected from a depth of 0.4m and the other half were collected from 1m depth. The sediments were sieved through a 250 µm mesh net and retained in buckets. Since, by definition, a “macroinvertebrate is any invertebrate that is 250 µm or larger,” the mesh net effectively eliminated macroinvertebrates from our sediment samples. The CPOM retained by the net was placed into 1 liter bottles. The retained CPOM would later be used to to determine the CPOM density of Lancer Park Pond and to give an estimate of the amount of CPOM to add to our BOD bottles. Water was collected from the pond at a depth of 0.5m on June 9, 2014.
    
