\section{Methods}
\subsection{Study Site}
All of the ponds used in the study are located in the vicinity of Farmville, VA (37.301 N, -78.396 W) and are small man--made ponds (Table \ref{tab:study_ponds}).  Lancer Park Pond is a 0.06 ha in--line pond with an earth dam and a maximum depth of 1.5 m. Lancer Park Pond has a permanent inlet and is almost completely surrounded by second growth forest. Daulton Pond is a 0.6 ha headwater pond with a earth dam and a maximum depth of 3.2 m. Daulton Pond does not have a permanent inlet and is likely partially spring--fed. The riparian zone of Daulton Pond is approximately 50\% second growth forest and 50\% mowed grass. Daulton Pond is the only pond in the study that has a substantial community of macrophytes. The pond is partially fringed by a population of cattails (\emph{Typha sp.}) that cover approximately 30\% of its littoral zone, and a bed of MACROPHYTE along its southern shore. Woodland Court Pond is a 0.3 ha pond with an earth dam that is drained by a stand--pipe. The pond has a maximum depth of INSERT DEPTH and a riparan zone that is about 30\% second growth forest. The remaining portion of the riparian zone is minimally landscaped disturbed land associated with an apartment complex. Wilkes Lake is a 


    
