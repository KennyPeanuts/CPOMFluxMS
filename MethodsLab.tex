\subsection{Incubation of Unleached Litter}
We evaluated the effect of leaf litter on sediment oxygen demand and nutrient flux using laboratory microcosms made from 300 ml BOD bottles. The experiment consisted of two treatments crossed in a randomized factorial design, with each treatment combination replicated 4 times. The treatment combinations were sediments only with ambient nutrients, sediments plus leaf litter with ambient nutrients, sediments only plus elevated dissolved inorganic phosphorus (DIP) and elevated dissolved inorganic nitrogen (DIN), and sediment plus leaf litter with elevated DIP and DIN. 

To set up the microcosms, sediments and pond water from Lancer Park Pond near the littoral zone (approximately 10 m from shore) on May 29, 2014, using an Ekman dredge.  Two dredge samples were collected from a depth of 0.4 m and another 2 dredge samples were collected from a depth of 1 m. The sediments from each location were combined and sieved through a 250 µm mesh net to remove macroinvertebrates and CPOM. The material retained by the nets was dried at 50\textsuperscript{o} C and ashed at 550\textsuperscript{o} C for 4 h to determine AFDM of the CPOM in the pond. The water for the microcosms was  collected from the pond at 0.5 m on June 9, 2014 by submerging 1 L plastic bottles.

The sieved sediment slurry was homogenized and allowed to settle overnight. The overlaying water was siphoned off and 100 ml of sediment was added to each of the 300 ml BOD bottles. The bottles were then carefully filled with approximately 200 ml of pond water, taking care not to disturb the sediments. Five 10 ml samples of the sediment slurry were collected and placed into pre-weighed crucibles, dried at 50\textsuperscript{o} C, and then ashed at 550\textsuperscript{o} C. to estimate the organic matter content of the sediments in the microcosms.

The organic matter content of the leaf disks was determined by randomly selecting 5 senescent tulip poplar leaves that were collected from fall 2013. Each leaf was gently submerged in DI water until it was soft enough to core (about 5 min). A single leaf disk (10mm, #5 cork borer) was cut from the leaf blade avoiding the midrib. The disk was placed into a pre-weighed crucible and dried at 50 degrees Celcius, then ashed at 550 degrees Celcius. CPOM treated BOD bottles ultimately recieved 20 leaf disks.

An initial water analysis was performed on the collected water to give a nutrient baseline for the ambient replacement water. Dr. Dina Leech filtered 50 ml of the collected pond water through the GFF and utilized Hach test kits to measure nitrate, nitrite, ammonia, and orthophosphate levels in the ambient water. N was tested using Hach Test Kit N1-12 (cat# 14081-00) using the provided instructions. Detection limits were <8.8 mg/L* for  Nitrate, <0.066mg/L for Nitrite, <0.2 mg/L for Ammonia, <0.2 mg/L for Orthophosphate. The nutrient enriched water that was used for nutrient treatment BOD bottles had a target of 300 \mu m NH4NO3 and 30 \mu m of KH2PO4.

*This number needs to be corrected


    
    
