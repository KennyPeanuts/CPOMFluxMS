\subsection{Leaf Litter Incubations}
We conducted 3 microcosm incubation experiments in order to evaluate the impact of leaf litter on sediment metabolism and material cycling. In the first experiment (hereafter the "litter experiment"), we evaluated the effect of unleached leaf litter on sediment oxygen demand, nutrient flux, fungal abundance, organic matter C:N, and dissolved organic matter quality. In the second experiment (hereafter the "leached litter experiment"), we evaluated the effect of leached leaf litter on SOD, fungal abundance, organic matter C:N, and dissolved organic matter quality. We were unable to get usable data on SOD from the second experiment however because of some technical difficulties with the Winkler titrations, so we conducted a third experiment (hereafter the "leached litter SOD experiment"), that only evaluated the effect of leached leaf litter on SOD.

All three experiments used laboratory microcosms made from 300 ml BOD bottles. The litter experiment consisted of two treatments crossed in a randomized factorial design, with each treatment combination replicated 4 times. The treatment combinations were: sediments only with ambient nutrients, sediments plus leaf litter with ambient nutrients, sediments only plus elevated dissolved inorganic phosphorus (DIP) and elevated dissolved inorganic nitrogen (DIN), and sediment plus leaf litter with elevated DIP and DIN. The leached litter experiment consisted of one treatment with 3 levels each replicated 4 times: sediments only, leaf litter only, and sediment plus leaf litter.  The leached litter SOD experiment consisted of one treatment with two levels each replicated 5 times: sediment only and sediment plus leaf litter. 

In all three experiments the microcosms were set up similarly. Sediments were collected from Lancer Park Pond near the littoral zone (approximately 10 m from shore) using an Ekman dredge on May 29 2014 for the litter experiment, 18 September 2015 for the leached litter experiment, and on 29 January 2016 for the leached litter SOD experiment.  In all cases the collected sediments were combined and sieved through a 250 µm mesh net to remove macroinvertebrates and CPOM. In the litter and leached litter experiment we combined the sediments from 4 total Ekman samples and in the leached litter SOD experiment, we combined the sediments from 6 total Ekman samples. The material retained by the nets during the sampling for the litter experiment was dried at 50\textsuperscript{o} C and ashed at 550\textsuperscript{o} C for 4 h to determine the AFDM of the CPOM in the pond, however the CPOM collected on the other sampling dates was not kept. 

For the litter experiment, the sieved sediment slurry was homogenized and allowed to settle overnight. The overlaying water was siphoned off and 100 ml of sediment was added to each of the 300 ml BOD bottles. The bottles were then carefully filled with approximately 200 ml of pond water (with or without added nutrients, see below), taking care not to disturb the sediments. For the leached litter and leached litter SOD experiments, 100 ml and 150 ml, respectively of the sediment slurry was added to the bottles without allowing it to settle first. The bottles were then topped off with pond water and allowed to settle overnight. The overlying water in each of the bottles was then siphoned off and replaced with pond water. In the leached litter SOD experiment, the overlying water was replaced 2 times. The microcosms for the leaf litter only treatment of the leached litter experiment were simply filled with pond water.

For all three experiments pond water was collected near the center of the lake from just below the surface.  The pond water used for the litter experiment, leached litter experiment was collected on 9 June 2014 and 17 September 2015 respectively. The pond water for the leached litter SOD experiment was collected on 29 January 2016, 4 Febuary 2016, and 10 Febuary 2016. The water collected on 4 Febuary 2016 was used to replace some of the overlying water during the microcosm set--up but it was turbid due to a recent rain, so the water collected on 10 Febuary 2016 was used as the replacement water for all the sampling.

passed through a  on May 29, 2014, .  Two dredge samples were collected from a depth of 0.4 m and another 2 dredge samples were collected from a depth of 1 m. The sediments from each location were combined and sieved through a 250 µm mesh net to remove macroinvertebrates and CPOM. The material retained by the nets was dried at 50\textsuperscript{o} C and ashed at 550\textsuperscript{o} C for 4 h to determine the AFDM of the CPOM in the pond. The water for the microcosms was  collected from the pond at 0.5 m on June 9, 2014 by submerging 1 L plastic bottles.

The sieved sediment slurry was homogenized and allowed to settle overnight. The overlaying water was siphoned off and 100 ml of sediment was added to each of the 300 ml BOD bottles. The bottles were then carefully filled with approximately 200 ml of pond water (with or without added nutrients, see below), taking care not to disturb the sediments. 

The microcosms were randomly assigned treatment combinations. To create microcosms with added leaf litter, we added 20, 10 mm disks punched from senescent tulip poplar leaves collected in the fall of 2013 (collection described above) with a \# 5 cork borer, to randomly selected microcosms. Prior to punching the leaf discs the leaves were softened in deionized water for approximately 5 minutes. Nutrient enriched treatments were created by enriching the pond water added to those bottles with 300 $\mu$g L\textsuperscript{-1} of DIN in the form of NH\textsubscript{4}NO\textsubscript{3} and 30 $\mu$g L\textsuperscript{-1} of DIP in the form of PO\textsubscript{4}.

The microcosms were incubated in the dark, at room temperature and agitated gently on a rocker-shaker. Between sampling (described below) 15 ml of the overlying water was removed from the bottles and the tops were removed to permit gas exchange.

The organic matter content of the microcosms was estimated from 5, 10 ml samples of the sediment slurry, and 5 randomly selected leaf discs. These organic matter samples were placed into pre-weighed crucibles, dried at 50\textsuperscript{o} C, and then ashed at 550\textsuperscript{o} C to determine the mass of organic matter.



    
    
