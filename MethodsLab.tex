\subsection{Incubation of Unleached Litter}
We evaluated the effect of leaf litter on sediment oxygen demand and nutrient flux using laboratory microcosms made from 300 ml BOD bottles. The experiment consisted of two treatments crossed in a randomized factorial design, with each treatment combination replicated 4 times. The treatment combinations were sediments only with ambient nutrients, sediments plus leaf litter with ambient nutrients, sediments only plus elevated dissolved inorganic phosphorus (DIP) and elevated dissolved inorganic nitrogen (DIN), and sediment plus leaf litter with elevated DIP and DIN. 

To set up the microcosms, we collected sediments from Lancer Park Pond near the littoral zone (approximately 10 m from shore) on May 29, 2014, using an Ekman dredge.  Two dredge samples were collected from a depth of 0.4 m and another 2 dredge samples were collected from a depth of 1 m. The sediments from each location were combined and sieved through a 250 µm mesh net to remove macroinvertebrates and CPOM. The material retained by the nets was dried at 50\textsuperscript{o} C and ashed at 550\textsuperscript{o} C for 4 h to determine the AFDM of the CPOM in the pond. The water for the microcosms was  collected from the pond at 0.5 m on June 9, 2014 by submerging 1 L plastic bottles.

The sieved sediment slurry was homogenized and allowed to settle overnight. The overlaying water was siphoned off and 100 ml of sediment was added to each of the 300 ml BOD bottles. The bottles were then carefully filled with approximately 200 ml of pond water (with or without added nutrients, see below), taking care not to disturb the sediments. 

The microcosms were randomly assigned treatment combinations. To create microcosms with added leaf litter, we added 20, 10 mm disks punched from senescent tulip poplar leaves collected in the fall of 2013 (collection described above) with a \# 5 cork borer, to randomly selected microcosms. Prior to punching the leaf discs the leaves were softened in deionized water for approximately 5 minutes. Nutrient enriched treatments were created by enriching the pond water added to those bottles with 300 $\mu$g L\textsuperscript{-1} of DIN in the form of NH\textsubscript{4}NO\textsubscript{3} and 30 $\mu$g L\textsuperscript{-1} of DIP in the form of PO\textsubscript{4}.

The microcosms were incubated in the dark, at room temperature and agitated gently on a rocker-shaker. Between sampling (described below) 15 ml of the overlying water was removed from the bottles and the tops were removed to permit gas exchange.

The organic matter content of the microcosms was estimated from 5, 10 ml samples of the sediment slurry, and 5 randomly selected leaf discs. These organic matter samples were placed into pre-weighed crucibles, dried at 50\textsuperscript{o} C, and then ashed at 550\textsuperscript{o} C to determine the mass of organic matter.



    
    
