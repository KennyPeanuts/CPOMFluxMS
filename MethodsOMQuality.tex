\subsection{Dissolved Organic Matter Quality}

The lability of the dissolved organic matter in water overlying the sediments in the bottles was estimated by evaluating the absorption spectra a 5 ml sample of the overlying water collected during each sampling period. The sample was collected from approximately halfway down the water column with a plastic syringe and a steel cannula. The sample was then filtered through a GFF filter (nominal pore size = 0.7 $\mu m$) and stored at 4 \textsuperscript{o}C. Absorbance spectra were determined within 24 h using a Nanopore spectrophotometer with a 1 cm path length. All spectra were baseline corrected by subtracting the absorption at 700 nm and converted to an absorption coefficient ($a$) using $a = \frac{2.303A}{I}$ where A is the raw absorption, and I is the path length in m.

We assessed the lability of the dissolved organic matter in the overlying water using the slope ratio (S\textsubscript{R}) \cite{helmsetal2008}. Briefly S\textsubscript{R} is the ratio of the slope of the natural log transformed $a$ by wavelength between 275 -- 295 nm and 350 -- 400 nm. S\textsubscript{R} has been shown to be greater when the molecular weight of the dissolved organic matter is greater \cite{helmsetal2008}. 

  