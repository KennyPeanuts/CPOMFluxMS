\subsection{Leaf Litter Incubation Experiments}
\subsubsection{Organic Matter}
The sediments added to the microcosms in the litter experiment had a dry bulk density that ranged between 0.25 to 0.26 g ml\textsuperscript{-1} thus we added between 25 and 26 g dry mass of sediment to the bottles. The organic matter content of those sediments ranged between 12.7 and 12.9\% so the sediments in the microcosms contained between 3.18 and 3.23 g of organic matter each.  The microcosms that received leaf litter each received between 0.04 and 0.08 additional g of organic matter with the added leaf litter, which would be the equivalent of 11.3 to 22.7 g coarse organic matter m\textsuperscript{-2}. 



The percent carbon of the surface sediments in the litter experiment ranged between 3.05\% and 3.41\% with a median of 3.18\%.  The bulk sediments ranged between 3.12 and 4.27\% C with a median of 3.29 \%.  The difference between the bulk sediment \% carbon and the surface sediment \% carbon was small with a range of --0.13 to 1.13\% and a median of 0.09\%. Only 3 samples departed more than 0.5\% and in all of these the bulk sediment had greater \% C. The bulk and surface sediment \% C of the same bottle were uncorrelated (r = -0.12) and the leaf litter or nutrient additions had no effect (p >0.05) on the \% C of either the surface or bulk sediment.

The \% N in the sediments was very highly correlated with the \% C of the same sediment sample (r = 0.968) and therefore, the \% N of the bulk and surface sediments showed the same patterns as the \% C.  The \% N of the surface sediments ranged between 0.28 and 0.32\% with a median of 0.29\%. The \% N of the bulk sediments  (median = 0.31\%, range = 0.29 to 0.40\%) was overall slightly greater than the surface sediments with a median difference of 0.01\% and only 2 samples out of 14 with greater \% N in the surface sediment.  As with the \% C, there was no effect (p > 0.05) of the leaf litter or nutrient additions on the \% N of either the surface or bulk sediments. 

The C:N of the sediments was constrained and only ranged between 10.48 and 11.27 across all of the bulk and surface sediment samples. There was no effect (p > 0.05) of added leaf litter or nutrients on the C:N of either the bulk or surface sediments. The C:N of the added leaf litter ranged 15.22 to 17.36 with a median of 16.16 which was significantly greater than the median C:N of either the surface or bulk sediments (p < 0.0001) and was not affected by the nutrient addition (p = 0.248). 

The sediments added to the microcosms in the leached litter experiment had a dry bulk density between 0.10 and 0.11 g ml\textsuperscript{-1}, which was less than those used in the leaf experiment and likely due to the fact that the sediment slurry was not settled before being added to the microcosm. The organic matter content of the sediments used in the leached litter experiment was nearly identical to those used in the litter experiment and ranged between 12.1 and 12.4\%. As a result the sediments of each microcosm contained between 1.23 and 1.39 g of organic matter. The microcosms that received leaf litter each received between 0.04 and 0.05 g of coarse organic matter with the added litter, which would be the equivalent of 10.8 to 13.5 g of coarse organic matter m\textsuperscript{-2}.

At the conclusion of the leached litter experiment (day ##) the sediments in the microcosms without leaf litter had a median of 13.3\% organic matter (range = 13.1\% -- 28.5\%) and the sediments in the microcosms with leaf litter had a median of 12.4\% organic matter (range = 11.8\% -- 14.3\%). The leaves in the microcosms containing leaves lost a mean ($\pm$ SD) of 30.7\% ($\pm$ 11.7) and 45.8\% ($\pm$ 8.1) of the initial organic matter mass of the leaves in the leaf--only and leaf + sediments treatments respectively. The difference in the percent of organic matter mass lost by the leaves was marginally significantly different between the microcosms with and without sediments (F = 4.936, df\textsubscript{error} = 6, p = 0.06804).

The sediments added to the microcosms in the leached litter SOD experiment had  median dry bulk density of 0.17 g ml\textsuperscript{-1}. The organic matter content of the sediments was more variable than was seen in the previous experiments, where one of the samples had only 7.2\% organic matter and the other 2 samples had 16.7 and 16.5\% organic matter. The sediments in this experiment contained between 4.07 and 4.24 g of organic matter per microcosm. The microcosms that received leaf litter each received between 0.04 and 0.07 g of coarse organic matter which would be the equivalent of 11.5 to 21.0 g of coarse organic matter m\textsuperscript{-2}.

In the lake from which the sediments were collected we measured between 57.3 and 163.0 g coarse organic matter m\textsuperscript{-2}, so the microcosms used in the experiment contained at most 40\% of the coarse organic matter of the lake. However in the microcosms all of the leaf litter was added to the surface of the sediments, while a portion of the leaf litter in the lakes was buried within the sediment.

In the litter experiment the dissolved oxygen concentration of the overlying water in the bottles ranged from 80.09 to 275.80 $\mu mols$ O\textsubscript{2} L\textsuperscript{-1} with a median of 224.80 $\mu mols$ O\textsubscript{2} L\textsuperscript{-1}.  The dissolved oxygen concentration varied significantly over the course of the incubation (p < 0.0001) and was the lowest on the initial day of the incubation (Fig. \ref{fig:DO_days}). Additionally, the bottles containing leaf litter had significantly lower dissolved oxygen concentrations (p = 0.002) but this was only evident on day 0 and day 2 of the incubation (Fig. \ref{fig:DO_days}).

During the litter experiment the temperature ranged between 23.4 and 25.5\textsuperscript{o} C and the oxygen consumption of the bottles ranged between 357 and 2098 $\mu mols$ O\textsubscript{2} m\textsuperscript{-2} h\textsuperscript{-1}, with a median flux of 920 $\mu mols$ O\textsubscript{2} m\textsuperscript{-2} h\textsuperscript{-1}. Since the treatments with added leaf litter had more organic matter in the sediments than those without leaf litter, we normalized the oxygen flux to the estimated organic matter content of the sediments in the bottles. The organic matter normalized oxygen flux ranged from 7.34 to 42.54 $\mu mols$ O\textsubscript{2} (g organic matter)\textsuperscript{-1} h\textsuperscript{-1}, with a median flux of 18.97 $\mu mols$ O\textsubscript{2} (g organic matter)\textsuperscript{-1} h\textsuperscript{-1}.

The only factor that affected organic matter normalized oxygen flux in the litter experiment was the presence of leaf litter in the bottles (Table \ref{tab:SOD_ANOVA}). Bottles containing leaf litter had a mean oxygen flux of 23.85 $\mu mols$ O\textsubscript{2} (g organic matter)\textsuperscript{-1} h\textsuperscript{-1}, which was significantly greater than the mean oxygen flux of 17.67 $\mu mols$ O\textsubscript{2} (g organic matter)\textsuperscript{-1} h\textsuperscript{-1} in the bottles without leaf litter (Fig. \ref{fig:sod}). There was no significant interaction between the presence of leaf litter and the duration of the experiment (Table \ref{tab:SOD_ANOVA}) but the lack of any difference in oxygen flux on day 21 (Fig \ref{fig:sod}) might suggest that the difference in oxygen flux between the bottles with and without leaf litter was temporary. 

The oxygen consumption of the overlying water alone ranged from -3.5 to 3.1 $\mu mols$ L\textsuperscript{-1} h\textsuperscript{-1}, which translated to an areal flux ranging from -1.9 to 1.7 $\mu mols$ m\textsuperscript{-2} h\textsuperscript{-1}. A negative flux indicates that the oxygen concentration of the water increased during the incubation. Overall the oxygen consumption of the water did not contribute substantially to the total oxygen consumption of the bottles. The maximum oxygen consumption of the water was only 0.5\% of the minimum total oxygen consumption.  




    