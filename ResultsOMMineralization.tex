\subsection{Unleached Litter Experiment}
The sediments added to the mesocosms had an estimated dry bulk density that ranged between 0.25 to 0.26 g ml\textsuperscript{-1} and an organic matter content between 12.7 and 12.9\%. Using this information, we estimate that the sediments in the mesocosms contained between 3.16 and 3.33 g of organic matter.  The mesocosms that received leaf litter each received between 0.04 and 0.08 additional g of organic matter with the added leaf litter, which would be the equivelent of 11.3 to 22.7 g coarse organic matter m\textsuperscript{-2}. In the lake from which the sediments were collected we measured between 57.3 and 163.0 g coarse organic matter m\textsuperscript{-2}, so the mesocosms used in the experiment contained at most 40\% of the coarse organic matter of the lake. However in the mesocosms all of the leaf litter was added to the surface of the sediments, while a portion of the leaf litter in the lakes was buried within the sediment.

The dissolved oxygen concentration of the overlying water in the bottles ranged from 80.09 to 275.80 $\mu mols$ O\textsubscript{2} L\textsubscript{-1} with a median of 224.80 $\mu mols$ O\textsubscript{2} L\textsuperscript{-1}.  The dissolved oxygen concentration varied significantly over the course of the incubation (p < 0.0001) and was the lowest on the initial day of the incubation (Fig. \ref{fig:DO_days}). Additionally, the bottles containing leaf litter had significantly lower dissolved oxygen concentrations (p = 0.002) but this was only evident on day 0 and day 2 of the incubation (Fig. \ref{fig:DO_days}).

Over the course of the 21 days of incubation the temperature ranged between 23.4 and 25.5\textsuperscript{o} C and the oxygen consumption of the bottles ranged between 357 and 2098 $\mu mols$ O\textsubscript{2} m\textsuperscript{-2} h\textsuperscript{-1}, with a median flux of 920 $\mu mols$ O\textsubscript{2} m\textsuperscript{-2} h\textsuperscript{-1}. Since the treatments with added leaf litter had more organic matter in the sediments than those without leaf litter, we normalized the oxygen flux to the estimated organic matter content of the sediments in the bottles. The organic matter normalized oxygen flux ranged from 7.34 to 42.54 $\mu mols$ O\textsubscript{2} (g organic matter)\textsuperscript{-1} h\textsuperscript{-1}, with a median flux of 18.97 $\mu mols$ O\textsubscript{2} (g organic matter)\textsuperscript{-1} h\textsuperscript{-1}.

The only factor that affected organic matter normalized oxygen flux was the presence of leaf litter in the bottles (Table \ref{tab:SOD_ANOVA}). Bottles containing leaf litter had a mean oxygen flux of 23.85 $\mu mols$ O\textsubscript{2} (g organic matter)\textsuperscript{-1} h\textsuperscript{-1}, which was significantly greater than the mean oxygen flux of 17.67 $\mu mols$ O\textsubscript{2} (g organic matter)\textsuperscript{-1} h\textsuperscript{-1} in the bottles without leaf litter (Fig. \ref{fig:sod}). There was no significant interaction between the presence of leaf litter and the duration of the experiment (Table \ref{tab:SOD_ANOVA}) but the lack of any difference in oxygen flux on day 21 (Fig \ref{fig:sod}) might suggest that the difference in oxygen flux between the bottles with and without leaf litter was temporary. 

The oxygen consumption of the overlying water alone ranged from -3.5 to 3.1 $\mu mols$ L\textsuperscript{-1} h\textsuperscript{-1}, which translated to an areal flux ranging from -1.9 to 1.7 $\mu mols$ m\textsuperscript{-2} h\textsuperscript{-1}. A negative flux indicates that the oxygen concentration of the water increased during the incubation. Overall the oxygen consumption of the water did not contribute substantially to the total oxygen consumption of the bottles. The maximum oxygen consumption of the water was only 0.5\% of the minimum total oxygen consumption.  




    